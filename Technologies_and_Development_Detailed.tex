\documentclass[12pt,a4paper]{article}
\usepackage[utf8]{inputenc}
\usepackage[french]{babel}
\usepackage{geometry}
\usepackage{fancyhdr}
\usepackage{titlesec}
\usepackage{listings}
\usepackage{xcolor}
\usepackage{hyperref}
\usepackage{graphicx}

\geometry{margin=2.5cm}
\pagestyle{fancy}
\fancyhf{}
\fancyhead[L]{Technologies et Développement - EMSI Share-Learn}
\fancyfoot[C]{\thepage}

% Code styling
\definecolor{codegreen}{rgb}{0,0.6,0}
\definecolor{codegray}{rgb}{0.5,0.5,0.5}
\definecolor{codepurple}{rgb}{0.58,0,0.82}
\definecolor{backcolour}{rgb}{0.95,0.95,0.92}

\lstdefinestyle{mystyle}{
    backgroundcolor=\color{backcolour},   
    commentstyle=\color{codegreen},
    keywordstyle=\color{magenta},
    numberstyle=\tiny\color{codegray},
    stringstyle=\color{codepurple},
    basicstyle=\ttfamily\footnotesize,
    breakatwhitespace=false,         
    breaklines=true,                 
    captionpos=b,                    
    keepspaces=true,                 
    numbers=left,                    
    numbersep=5pt,                  
    showspaces=false,                
    showstringspaces=false,
    showtabs=false,                  
    tabsize=2
}

\lstset{style=mystyle}

\title{\textbf{Technologies et Développement\\EMSI Share-Learn}}
\author{Projet de Fin d'Année}
\date{\today}

\begin{document}

\maketitle
\tableofcontents
\newpage

\section{Développement Frontend}

\subsection{Structure générale et Configuration}

Le frontend du projet EMSI Share-Learn est développé avec \textbf{React 18.2.0} et utilise \textbf{TypeScript 5.2.2} pour assurer un typage robuste et une meilleure maintenance du code. Le projet est configuré avec \textbf{Vite 6.3.5} comme outil de build moderne, offrant un développement rapide grâce à son système de Hot Module Replacement (HMR) et des temps de compilation optimisés.

La configuration Vite est personnalisée pour supporter :
\begin{itemize}
    \item Un serveur de développement sur le port 8080 avec accès réseau (host: "0.0.0.0")
    \item Des alias de chemins avec "@" pointant vers "./src"
    \item L'intégration complète avec React et TypeScript
\end{itemize}

\subsection{Gestion du Design et Styling}

Le design est entièrement géré avec \textbf{Tailwind CSS 3.4.1}, adoptant une approche utilitaire moderne qui facilite la gestion cohérente du style. La configuration Tailwind inclut :

\textbf{Palette de couleurs personnalisée EMSI :}
\begin{itemize}
    \item Couleur primaire : \texttt{\#00a651} (vert EMSI Tanger)
    \item Variantes : \texttt{\#5dbea3} (vert clair), \texttt{\#00853e} (vert foncé)
    \item Couleurs secondaires : slate blue-gray, amber, et une palette complète pour les modes sombre/clair
\end{itemize}

\textbf{Système de thèmes avancé :}
\begin{itemize}
    \item Support complet du mode sombre/clair avec \texttt{next-themes 0.3.0}
    \item Variables CSS personnalisées pour une cohérence visuelle
    \item Animations et transitions fluides avec \texttt{tailwindcss-animate}
\end{itemize}

\subsection{Architecture et Organisation du Code}

Le code frontend suit une architecture modulaire stricte organisée en plusieurs dossiers spécialisés :

\textbf{Structure des composants :}
\begin{lstlisting}
src/
├── components/
│   ├── admin/           # Composants d'administration (PendingResourcesPanel)
│   ├── dashboard/       # Tableaux de bord (StudentDashboard, TeacherDashboard)
│   ├── editor/          # Éditeurs (CodeEditor, MarkdownEditor)
│   ├── events/          # Gestion d'événements (CreateEventDialog, CollaboratorsList)
│   ├── forum/           # Forum (TopicDetail, NewDiscussionModal, RichTextEditor)
│   ├── layout/          # Mise en page (Header, Sidebar, MainLayout)
│   ├── resources/       # Ressources (ResourceCard, PDFViewer, ResourceUploadDialog)
│   ├── rooms/           # Salles de classe (RoomCard, CreateRoomDialog)
│   └── ui/              # Composants UI réutilisables (50+ composants)
\end{lstlisting}

\textbf{Pages principales :}
\begin{itemize}
    \item \texttt{Dashboard.tsx} : Tableau de bord principal avec logique métier
    \item \texttt{Resources.tsx} : Gestion complète des ressources pédagogiques
    \item \texttt{Forum.tsx} : Système de forum avec discussions avancées
    \item \texttt{Events.tsx} : Planification et gestion d'événements
    \item \texttt{Quiz.tsx} : Interface de quiz interactifs
    \item \texttt{Rooms.tsx} / \texttt{StudentRooms.tsx} : Gestion des salles de classe
\end{itemize}

\subsection{Gestion d'État et Contextes}

L'application utilise plusieurs contextes React pour la gestion d'état globale :

\textbf{AuthContext :} Gestion complète de l'authentification avec :
\begin{itemize}
    \item Authentification JWT avec cookies sécurisés
    \item Gestion des rôles utilisateur (student, teacher, admin, administration)
    \item Mise à jour de profil en temps réel
    \item Gestion des photos de profil avec stockage binaire
\end{itemize}

\textbf{ThemeContext :} Système de thèmes avec :
\begin{itemize}
    \item Basculement automatique sombre/clair
    \item Persistance des préférences utilisateur
    \item Synchronisation avec les variables CSS Tailwind
\end{itemize}

\textbf{PlatformContext :} Configuration de la plateforme avec :
\begin{itemize}
    \item Paramètres globaux personnalisables
    \item Gestion du logo et branding
    \item Configuration des tailles de pagination
\end{itemize}

\subsection{Bibliothèques et Dépendances Avancées}

\textbf{Interface utilisateur (Radix UI) :}
\begin{itemize}
    \item 15+ composants Radix UI pour une accessibilité maximale
    \item Composants complexes : Dialog, Dropdown, Select, Toast, Tooltip
    \item Gestion avancée des états et interactions
\end{itemize}

\textbf{Gestion des données :}
\begin{itemize}
    \item \texttt{@tanstack/react-query 5.28.4} : Cache intelligent et synchronisation serveur
    \item Gestion optimisée des requêtes API avec invalidation automatique
    \item États de chargement et d'erreur centralisés
\end{itemize}

\textbf{Traitement de documents :}
\begin{itemize}
    \item \texttt{@react-pdf-viewer/core 3.12.0} : Visualisation PDF avancée
    \item \texttt{react-pdf 7.7.1} : Rendu PDF côté client
    \item \texttt{docx 9.5.0} : Traitement des documents Word
    \item \texttt{xlsx 0.18.5} : Gestion des fichiers Excel
    \item \texttt{jszip 3.10.1} : Manipulation d'archives ZIP
\end{itemize}

\textbf{Éditeurs et formatage :}
\begin{itemize}
    \item \texttt{react-markdown 10.1.0} : Rendu Markdown avec support LaTeX
    \item \texttt{react-katex 3.1.0} : Formules mathématiques
    \item \texttt{highlight.js 11.11.1} : Coloration syntaxique du code
    \item \texttt{prism-react-renderer 2.4.1} : Mise en évidence de code avancée
\end{itemize}

\subsection{Services API et Communication Backend}

Le système de communication avec le backend est centralisé dans \texttt{src/services/api.ts} avec :

\textbf{Architecture API modulaire :}
\begin{lstlisting}[language=JavaScript]
// Modules API spécialisés
export const authAPI = { login, register, getMe, updateProfile }
export const roomsAPI = { getRooms, createRoom, joinRoom, getTeacherStudents }
export const resourcesAPI = { getResources, uploadResource, downloadResource }
export const forumsAPI = { getTopics, createTopic, getPosts, createPost }
export const eventsAPI = { getEvents, createEvent, attendEvent }
export const quizzesAPI = { getQuizzes, submitQuiz, getResults }
export const notificationsAPI = { getNotifications, markAsRead }
\end{lstlisting}

\textbf{Gestion d'erreurs robuste :}
\begin{itemize}
    \item Retry automatique pour les erreurs de connexion PostgreSQL
    \item Gestion spécifique des contraintes de base de données
    \item Messages d'erreur contextuels et informatifs
    \item Fallback gracieux en cas d'échec réseau
\end{itemize}

\textbf{Authentification sécurisée :}
\begin{itemize}
    \item Tokens JWT avec refresh automatique
    \item Cookies sécurisés avec expiration configurée
    \item Headers d'autorisation automatiques
    \item Gestion des sessions expirées
\end{itemize}

\subsection{Fonctionnalités Frontend Avancées}

\textbf{Upload de fichiers sophistiqué :}
\begin{itemize}
    \item Support multi-format (PDF, Word, Excel, ZIP, images, vidéos)
    \item Validation côté client avec preview
    \item Barre de progression en temps réel
    \item Gestion des gros fichiers (jusqu'à 500MB)
\end{itemize}

\textbf{Interface responsive complète :}
\begin{itemize}
    \item Design mobile-first avec Tailwind CSS
    \item Breakpoints personnalisés : sm, md, lg, xl, 2xl
    \item Composants adaptatifs pour tous les écrans
    \item Navigation optimisée tactile
\end{itemize}

\textbf{Accessibilité (WCAG 2.1) :}
\begin{itemize}
    \item Support complet du clavier
    \item Lecteurs d'écran compatibles
    \item Contraste de couleurs optimisé
    \item Focus management avancé
\end{itemize}

\section{Développement Backend}

\subsection{Architecture Django et Configuration}

Le backend est développé avec \textbf{Django 5.0.0}, un framework Python robuste, associé à \textbf{Django REST Framework 3.14.0} pour la création d'APIs REST modernes. La structure suit les meilleures pratiques Django avec une organisation modulaire par applications.

\textbf{Configuration de base :}
\begin{lstlisting}[language=Python]
# Applications Django spécialisées
INSTALLED_APPS = [
    'users',              # Gestion utilisateurs avancée
    'rooms',              # Salles de classe virtuelles
    'resources',          # Ressources pédagogiques
    'quizzes',            # Système de quiz
    'forums',             # Forum de discussion
    'notifications',      # Notifications en temps réel
    'events',             # Gestion d'événements
    'platform_settings',  # Configuration plateforme
]
\end{lstlisting}

\subsection{Base de Données PostgreSQL}

\textbf{Configuration PostgreSQL optimisée :}
\begin{itemize}
    \item Base de données : \texttt{emsi\_share\_db}
    \item Utilisateur : \texttt{postgres} avec permissions complètes
    \item Port : 5432 (configuration standard)
    \item Gestion des gros objets avec \texttt{large\_objects\_config.sql}
\end{itemize}

\textbf{Limites de fichiers configurées :}
\begin{itemize}
    \item \texttt{DATA\_UPLOAD\_MAX\_MEMORY\_SIZE} : 500MB
    \item \texttt{FILE\_UPLOAD\_MAX\_MEMORY\_SIZE} : 500MB
    \item Support des fichiers binaires en base avec \texttt{BinaryField}
\end{itemize}

\subsection{Modélisation Avancée des Données}

\textbf{Modèle User étendu :}
\begin{lstlisting}[language=Python]
class User(AbstractUser):
    ROLE_CHOICES = [
        ('student', 'Student'),
        ('teacher', 'Teacher'), 
        ('admin', 'Admin'),
        ('administration', 'Administration'),
    ]
    email = models.EmailField(unique=True)  # Email comme identifiant
    role = models.CharField(max_length=15, choices=ROLE_CHOICES)
    avatar = models.ImageField(upload_to='avatars/')
    profile_picture = models.BinaryField()  # Stockage binaire direct
    is_verified = models.BooleanField(default=False)
    last_activity = models.DateTimeField(auto_now=True)
\end{lstlisting}

\textbf{Modèle Resource sophistiqué :}
\begin{lstlisting}[language=Python]
class Resource(models.Model):
    RESOURCE_TYPES = [
        ('document', 'Document'), ('video', 'Video'), ('code', 'Code'),
        ('pdf', 'PDF Document'), ('audio', 'Audio'), ('image', 'Image'),
        ('doc', 'Word Document'), ('ppt', 'PowerPoint'), 
        ('excel', 'Excel'), ('zip', 'ZIP Archive'), ('other', 'Other'),
    ]
    STATUS_CHOICES = [
        ('pending', 'Pending Approval'),
        ('approved', 'Approved'), 
        ('rejected', 'Rejected'),
    ]
    file_data = models.BinaryField()  # Stockage binaire direct
    file_name = models.CharField(max_length=255)
    file_type = models.CharField(max_length=100)  # MIME type
    status = models.CharField(choices=STATUS_CHOICES, default='approved')
    reviewed_by = models.ForeignKey(User, related_name='reviewed_resources')
\end{lstlisting}

\textbf{Modèle Forum complet :}
\begin{lstlisting}[language=Python]
class ForumTopic(models.Model):
    STATUS_CHOICES = [('open', 'Open'), ('closed', 'Closed'), 
                     ('pinned', 'Pinned'), ('locked', 'Locked')]
    
    is_solved = models.BooleanField(default=False)
    solved_by = models.ForeignKey(User, related_name='solved_topics')
    view_count = models.IntegerField(default=0)
    like_count = models.IntegerField(default=0)
    attachment_data = models.BinaryField()  # Pièces jointes
    tags = models.CharField(max_length=500)  # Tags pour recherche
\end{lstlisting}

\subsection{API REST Framework Avancée}

\textbf{ViewSets spécialisés avec actions personnalisées :}

\textbf{ResourceViewSet :}
\begin{lstlisting}[language=Python]
@action(detail=True, methods=['post'], url_path='approve')
def approve_resource(self, request, pk=None):
    # Logique d'approbation avec notifications automatiques
    
@action(detail=True, methods=['get'], url_path='download') 
def download(self, request, pk=None):
    # Téléchargement sécurisé avec headers appropriés
\end{lstlisting}

\textbf{ForumTopicViewSet :}
\begin{lstlisting}[language=Python]
@action(detail=True, methods=['post'])
def toggle_solved(self, request, pk=None):
    # Marquage solution avec notification du solutionneur
    
@action(detail=True, methods=['post'])
def like_topic(self, request, pk=None):
    # Système de likes avec compteurs en temps réel
\end{lstlisting}

\subsection{Système d'Authentification JWT}

\textbf{Configuration Simple JWT :}
\begin{lstlisting}[language=Python]
SIMPLE_JWT = {
    'ACCESS_TOKEN_LIFETIME': timedelta(days=1),
    'REFRESH_TOKEN_LIFETIME': timedelta(days=14),
    'AUTH_COOKIE': 'emsi_access',
    'AUTH_COOKIE_REFRESH': 'emsi_refresh',
    'AUTH_COOKIE_HTTP_ONLY': True,
    'AUTH_COOKIE_SAMESITE': 'Lax',
}
\end{lstlisting}

\textbf{Endpoints d'authentification :}
\begin{itemize}
    \item \texttt{POST /api/token/} : Connexion avec email/mot de passe
    \item \texttt{POST /api/token/refresh/} : Renouvellement de token
    \item \texttt{POST /api/auth/register/} : Inscription avec validation
    \item \texttt{GET /api/auth/me/} : Profil utilisateur actuel
\end{itemize}

\subsection{Gestion des Permissions et Sécurité}

\textbf{Permissions personnalisées :}
\begin{lstlisting}[language=Python]
class IsOwnerOrReadOnly(permissions.BasePermission):
    def has_object_permission(self, request, view, obj):
        if request.method in permissions.SAFE_METHODS:
            return True
        return obj.uploaded_by == request.user
\end{lstlisting}

\textbf{Validation des rôles :}
\begin{itemize}
    \item Administrateurs : accès complet à toutes les ressources
    \item Enseignants : gestion de leurs salles et ressources
    \item Étudiants : soumission avec approbation requise
\end{itemize}

\subsection{Système de Notifications Avancé}

\textbf{Types de notifications automatiques :}
\begin{itemize}
    \item Approbation/rejet de ressources
    \item Nouvelles réponses dans le forum
    \item Invitations à des événements
    \item Rappels de quiz
    \item Activité dans les salles de classe
\end{itemize}

\textbf{Modèle Notification :}
\begin{lstlisting}[language=Python]
class Notification(models.Model):
    recipient = models.ForeignKey(User, related_name='notifications')
    sender = models.ForeignKey(User, related_name='sent_notifications')
    notification_type = models.ForeignKey(NotificationType)
    title = models.CharField(max_length=255)
    message = models.TextField()
    action_url = models.CharField(max_length=500)
    is_read = models.BooleanField(default=False)
    priority = models.CharField(choices=PRIORITY_CHOICES)
\end{lstlisting}

\subsection{Endpoints API Complets}

\textbf{Ressources :}
\begin{itemize}
    \item \texttt{GET /api/resources/} : Liste avec filtres (statut, catégorie, recherche)
    \item \texttt{POST /api/resources/} : Upload avec validation automatique
    \item \texttt{GET /api/resources/\{id\}/download/} : Téléchargement sécurisé
    \item \texttt{POST /api/resources/\{id\}/approve/} : Approbation (admin)
    \item \texttt{POST /api/resources/\{id\}/reject/} : Rejet avec raison
\end{itemize}

\textbf{Forum :}
\begin{itemize}
    \item \texttt{GET /api/forums/topics/} : Topics avec filtres avancés
    \item \texttt{POST /api/forums/topics/} : Création avec pièces jointes
    \item \texttt{POST /api/forums/topics/\{id\}/toggle\_solved/} : Marquage solution
    \item \texttt{GET /api/forums/posts/?topic=\{id\}} : Posts d'un topic
    \item \texttt{POST /api/forums/posts/\{id\}/vote/} : Système de votes
\end{itemize}

\textbf{Événements :}
\begin{itemize}
    \item \texttt{GET /api/events/} : Événements avec filtres temporels
    \item \texttt{POST /api/events/} : Création avec collaborateurs
    \item \texttt{POST /api/events/\{id\}/attend/} : Participation
    \item \texttt{GET /api/events/\{id\}/attendees/} : Liste des participants
\end{itemize}

\subsection{Gestion des Fichiers et Performance}

\textbf{Stockage binaire optimisé :}
\begin{itemize}
    \item Fichiers stockés directement en base (BinaryField)
    \item Compression automatique pour les gros fichiers
    \item Validation MIME type côté serveur
    \item Limitation de taille configurable
\end{itemize}

\textbf{Optimisations de requêtes :}
\begin{itemize}
    \item Select\_related et prefetch\_related systématiques
    \item Pagination automatique pour les grandes listes
    \item Index de base de données sur les champs de recherche
    \item Cache de requêtes fréquentes
\end{itemize}

\section{Conclusion}

Cette architecture complète et ces technologies modernes garantissent une plateforme robuste, scalable et maintenable pour l'écosystème éducatif EMSI.

\end{document}